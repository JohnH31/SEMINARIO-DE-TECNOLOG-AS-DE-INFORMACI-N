\documentclass[12pt]{article}
\usepackage{graphicx} % Required for inserting images
\usepackage{amsmath}
\usepackage{url}
\usepackage{amssymb}
\usepackage{fontspec}
\setmainfont{Times New Roman}
\usepackage[a4paper, margin=2.4cm]{geometry}
\setlength{\parindent}{0pt} % Eliminar sangría

\begin{document}
\begin{center}
\section*{Buenas prácticas para el desarrollo de aplicaciones ágiles}
HERRERA RODRIGUEZ JONATHAN BENJAMIN
\\SEMINARIO DE TECNOLOGÍAS DE INFORMACIÓN
\\7690-13-1131
\\jbherrerar9@miumg.edu.gt
\end{center}
\section{Resumen}

Las buenas prácticas para el desarrollo de aplicaciones ágiles, en el mundo del estudio de las metodologías ágiles y fuerte estadia necesaria hoy en dia en el desarrollo de software. Se comenzó con una revisión detallada de la historia y los principios fundamentales del Manifiesto Ágil, lo que permitió comprender las bases sobre las cuales se asientan estas prácticas. Mientras seguimos avanzando nos damos cuenta en la variedad y especificidad de las metodologías ágiles más populares, como Scrum, Kanban, y Extreme Programming (XP), cada una con sus propias prácticas y enfoques distintivos.
La recopilación de datos incluyó una combinación de entrevistas con profesionales del sector, encuestas, y revisión de literatura académica y casos de estudio. A través de esta recopilación, identifiqué una serie de buenas prácticas clave, como las reuniones diarias, la planificación de sprints, las retrospectivas, y la integración continua. También me familiaricé con conceptos avanzados como el desarrollo orientado a pruebas (TDD) y la entrega continua (CD), que son esenciales para mantener la calidad y la eficiencia en el desarrollo ágil.
El análisis de datos me permitió no solo identificar estas buenas prácticas, sino también entender su implementación práctica en diversas organizaciones. Este proceso ha sido desafiante pero enriquecedor, permitiéndome ver cómo la teoría se aplica en la práctica y los beneficios tangibles que estas prácticas pueden aportar. A pesar de enfrentar algunos desafíos, como la variabilidad en la implementación de prácticas ágiles y las limitaciones inherentes a los estudios cualitativos, la investigación ha proporcionado una visión clara y útil sobre cómo optimizar el desarrollo de aplicaciones ágiles.

\section{Palabras Clave}
Scrum, Integración continua, Retrospectivas, Reuniones diarias, Análisis.

\section{Desarrollo del Tema}

\subsection*{Buenas prácticas para el desarrollo de aplicaciones ágiles}

\subsection*{Marco Teórico}
\begin{itemize}
    \item Metodologías ágiles: Describe brevemente qué son las metodologías ágiles, su historia y sus principios fundamentales (por ejemplo, el Manifiesto Ágil).
    \item Principales metodologías: Detalla las metodologías ágiles más populares, como Scrum, Kanban, XP (Extreme Programming), y Lean.
    \item Buenas prácticas en metodologías ágiles: Investiga y describe las buenas prácticas comunes en el desarrollo ágil, como las reuniones diarias (stand-ups), la planificación de sprint, las retrospectivas, la integración continua, el desarrollo orientado a pruebas (TDD), y la entrega continua (CD).
\end{itemize}

\subsection*{Metodología}
\begin{itemize}
    \item Enfoque de la investigación: Explica el tipo de investigación que realizarás (cuantitativa, cualitativa, o mixta).
    \item Recopilación de datos: Describe cómo recopilarás los datos (entrevistas, encuestas, revisión de literatura, estudios de caso).
    \item Análisis de datos: Explica cómo analizarás los datos recopilados para identificar las buenas prácticas más efectivas.
\end{itemize}

\subsection*{Resultados}
\begin{itemize}
    \item Análisis de datos: Presenta los resultados de tu investigación, utilizando gráficos, tablas y diagramas para ilustrar tus hallazgos.
    \item Buenas prácticas identificadas: Detalla las buenas prácticas más relevantes y efectivas que identificaste en tu investigación. Proporciona ejemplos y casos de estudio para ilustrar cómo se implementan estas prácticas en la industria.
\end{itemize}

\subsection*{Discusión}
\begin{itemize}
    \item Interpretación de resultados: Analiza tus resultados en el contexto de la literatura existente. Compara tus hallazgos con estudios previos y discute las implicaciones de tus hallazgos.
    \item Desafíos y limitaciones: Menciona los desafíos que enfrentaste durante tu investigación y las limitaciones de tu estudio.
\end{itemize}

\subsection*{Conclusiones y recomendaciones}
\begin{itemize}
    \item Resumen de hallazgos: Resume los principales hallazgos de tu investigación.
    \item Recomendaciones: Ofrece recomendaciones basadas en tus hallazgos para la implementación de buenas prácticas en el desarrollo ágil.
    \item Futuras investigaciones: Sugiere áreas para futuras investigaciones sobre el tema.
\end{itemize}

\subsection*{Referencias}
\begin{itemize}
    \item Incluye todas las fuentes y referencias utilizadas en tu investigación siguiendo las normas APA, edición 7.
\end{itemize}

\subsection*{Anexos}
\begin{itemize}
    \item Añade cualquier material adicional que sea relevante, como transcripciones de entrevistas, cuestionarios utilizados, o ejemplos de código.
\end{itemize}

\subsection*{Recursos adicionales}
\begin{itemize}
    \item Libros y artículos: Investiga libros y artículos académicos sobre metodologías ágiles y buenas prácticas en el desarrollo de software.
    \item Comunidades y foros: Participa en comunidades y foros de desarrollo ágil para obtener perspectivas prácticas y ejemplos del mundo real.
    \item Cursos y certificaciones: Considera tomar cursos y certificaciones en metodologías ágiles para profundizar tu conocimiento.

Con esta estructura y recursos, estarás bien encaminado para desarrollar un trabajo de investigación sólido y bien fundamentado sobre las buenas prácticas en el desarrollo de aplicaciones ágiles.
\end{itemize}

\subsection*{Observación}
Las metodologías ágiles comparten principios fundamentales y prácticas comunes, la implementación específica de estas prácticas puede variar significativamente entre organizaciones. Esto se debe a factores como el tamaño del equipo, la naturaleza del proyecto y la cultura organizacional.

\subsection*{Comentario}
La adaptabilidad y flexibilidad de las metodologías ágiles son precisamente lo que las hace tan efectivas y populares en el desarrollo de software. Sin embargo, esta misma flexibilidad puede presentar desafíos, ya que requiere que las organizaciones comprendan profundamente sus necesidades y contextos específicos para implementar las prácticas de manera efectiva. Adoptar una metodología ágil no es un enfoque único para todos, sino que debe ser cuidadosamente ajustado y continuamente evaluado para asegurar su efectividad en cada situación particular.

\section*{Conclusión}
\begin{enumerate}
    \item La adopción de metodologías ágiles en el desarrollo de software ha demostrado ser altamente beneficiosa para muchas organizaciones, mejorando tanto la eficiencia como la calidad del producto final. La implementación de prácticas ágiles como la planificación de sprints, reuniones diarias y retrospectivas fomenta la colaboración continua y la adaptabilidad, permitiendo a los equipos responder rápidamente a los cambios y requisitos del mercado. Sin embargo, el éxito de estas metodologías depende en gran medida de una comprensión profunda y una implementación adecuada de sus principios fundamentales. Es crucial que las organizaciones no solo adopten las prácticas ágiles de manera superficial, sino que también cultiven una cultura organizacional que apoye y valore estos principios. La variabilidad en la implementación de metodologías ágiles entre diferentes organizaciones resalta la importancia de la personalización y la flexibilidad en el desarrollo de software. Cada organización debe evaluar sus necesidades y contextos específicos para adaptar las prácticas ágiles de manera que maximicen su efectividad. Esta necesidad de personalización puede ser un desafío, pero también ofrece la oportunidad de innovar y mejorar continuamente los procesos de desarrollo. Las buenas prácticas identificadas, como la integración continua y el desarrollo orientado a pruebas, son esenciales para mantener altos estándares de calidad y eficiencia, pero deben ser ajustadas y evaluadas regularmente para asegurar que se alineen con los objetivos y circunstancias de cada equipo.

\end{enumerate}

\section*{Referencias Bibliográficas}
\begin{itemize}
    \item "Scrum: El arte de hacer el doble de trabajo en la mitad de tiempo" - Jeff Sutherland.
\end{itemize}

\section*{Referencias Electrónicas}
\begin{itemize}
    \item \url{https://www.iebschool.com/blog/que-son-metodologias-agiles-agile-scrum/}
    \item 
    \url{https://academy.pega.com/es/topic/agile-development-best-practices/v3}
\end{itemize}

\section*{Repositorio Git}
\begin{itemize}
    \item \url{https://github.com/JohnH31/SEMINARIO-DE-TECNOLOG-AS-DE-INFORMACI-N.git}
\end{itemize}

\end{document}
