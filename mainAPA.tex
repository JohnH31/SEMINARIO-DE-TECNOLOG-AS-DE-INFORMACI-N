\documentclass[12pt]{article}
\usepackage{graphicx} % Required for inserting images
\usepackage{amsmath}
\usepackage{url}
\usepackage{amssymb}
\usepackage{fontspec}
\setmainfont{Times New Roman}
\usepackage[a4paper, margin=2.4cm]{geometry}
\setlength{\parindent}{0pt} % Eliminar sangría

\begin{document}
\begin{center}
\section*{Normas APA y Metodología de la Investigación}
HERRERA RODRIGUEZ JONATHAN BENJAMIN
\\SEMINARIO DE TECNOLOGÍAS DE INFORMACIÓN
\\7690-13-1131
\\jbherrerar9@miumg.edu.gt
\end{center}
\section*{Resumen}

Normas APA proporciona directrices estandarizadas para la presentación de trabajos académicos englobando la estructura, citas y referencia. Esto con el fin de facilitar la lectura y la comprensión de la documentación, asegurar la credibilidad para así evitar el plagio, mejorar la claridad y el profesionalismo en los documentos. Enfocándonos en la edición 7 nos deja colocar de manera concisa citas de múltiples autores, la presentación de referencias electrónicas y uso de lenguajes inclusivos y una estructura de documento actualizada. Enfocando en el formato del documento nos indica que lleva un margen predeterminado que hay que cumplir que es de 1 pulgada una fuente esta puede variar pero en este caso se usara Time New Roman 12 el espacio doble en hojas. El formato de las citas y referencias que hay que cumplir y el de las tablas.

Para usar esta metodología, se debe formular claramente el problema, revisar la literatura existente, diseñar la investigación seleccionando métodos cualitativos, cuantitativos o mixtos, recolectar y analizar los datos, interpretar los resultados, y presentar las conclusiones y recomendaciones. La hipótesis, planteada como una premisa razonada, guía el proceso de investigación, ordenando y sistematizando el conocimiento.

\section*{Palabras Clave}
Estandarización, Diversidad, Hipótesis, Análisis, Investigación.

\section*{Desarrollo del Tema}

\subsection*{Normas APA 7ma Edición}
Las Normas APA (American Psychological Association) son un conjunto de directrices para la presentación de trabajos escritos en el ámbito académico. Estas normas están enfocadas en aspectos como la estructura del documento, el formato de las citas y referencias, y la claridad de la escritura.

El propósito de las normas APA, estas tienen como objetivos: la estandarización esto hace que tenga un formato uniforme para la presentación del trabajo, para facilitar la lectura y la comprensión de los documentos académicos, la credibilidad para asegurar que se del crédito a las fuentes originales y para eso evitar el plagio y la claridad mejorar el profesionalismo de los trabajos escritos.

En su 7ma edición, se puede reflejar el formato de citas simplifica la forma de citar obras con múltiples autores, formato de referencias prestación de referencias electrónicas y digitales, inclusión de diversidad mayor énfasis en el uso del leguaje inclusivo y no discriminatorio, estructura del documento recomendaciones actualizados para la estructura y formato de los trabajos, incluyendo el uso de diferentes encabezados y niveles de títulos.

\subsection*{Formato del Documento}
\begin{itemize}
    \item Márgenes: 1 pulgada en todos los lados
    \item Fuente: Times New Roman 12 pt o una fuente legible similar
    \item Espaciado: Doble espacio en todo el documento
    \item Encabezados y Pies de Página: Número de página en la esquina superior derecha de cada página
\end{itemize}

\subsection*{Citas en el Texto}
\begin{itemize}
    \item Autor-fecha: (Apellido del autor, año)
    \item Citas directas: Incluir número de página (p. ej., (Apellido, año, p. X))
\end{itemize}

\subsection*{Referencias}
\begin{itemize}
    \item Libros: Apellido, N. (Año). Título del libro. Editorial
    \item Artículos: Apellido, N. (Año). Título del artículo. Nombre de la revista, volumen(número), páginas. DOI o URL si es aplicable
\end{itemize}

\subsection*{Ubicación de las Tablas}
Hay dos opciones para la ubicación de tablas (y figuras) en una investigación. La primera es incrustar tablas en el texto después de que la menciones por primera vez; la segunda opción es agregar cada tabla en una página separada después de la lista de referencias. Si la tabla es corta, puede mezclar en la misma página texto y la tabla. Intente agregar la tabla al comienzo o al final de la página. Ten en cuenta que una tabla también podría ocupar toda la página completa sin problema. Es posible que tengas que agregar una línea en blanco entre la tabla y el texto para mejorar la presentación visual.

\subsection*{Tablas Largas o Anchas}
Si una tabla es más larga que una página, puedes hacer con que la fila de encabezados se repita en la segunda página y en las páginas siguientes (cuantas veces sea necesario). Si una tabla es demasiado ancha para caber en una página, puedes utilizar orientación horizontal en la página con la tabla ancha.

\subsection*{Metodología de la Investigación}
Es el conjunto de procedimientos y técnicas utilizadas para recolectar, analizar y presentar datos en un estudio. Es un componente fundamental en la investigación científica, que asegura que los resultados sean válidos, confiables y replicables.

\subsection*{La Metodología de la Investigación se Usa para:}
\begin{itemize}
    \item Describir fenómenos: Permite la descripción detallada y sistemática de fenómenos, hechos o situaciones específicas
    \item Explicar relaciones: Ayuda a identificar y explicar las relaciones causales entre diferentes variables
    \item Predecir comportamientos: Facilita la predicción de comportamientos o eventos futuros basados en datos y patrones observados
    \item Desarrollar teorías: Contribuye al desarrollo de teorías y modelos que expliquen ciertos fenómenos
    \item Tomar decisiones: Proporciona una base sólida para la toma de decisiones informadas en diversos campos, como la política, la salud, la educación, y los negocios
\end{itemize}

\subsection*{Cómo Usar la Metodología de la Investigación:}
\begin{itemize}
    \item Formulación del problema: Definir claramente el problema o la pregunta de investigación
    \item Revisión de la literatura: Investigar estudios previos y teorías relacionadas con el tema
    \item Diseño de la investigación: Elegir el tipo de investigación (cualitativa, cuantitativa o mixta) y seleccionar los métodos de recolección de datos (encuestas, entrevistas, experimentos, etc.)
    \item Recolección de datos: Implementar las técnicas seleccionadas para recolectar los datos necesarios
    \item Análisis de datos: Utilizar métodos estadísticos o cualitativos para analizar los datos recolectados
    \item Interpretación de resultados: Interpretar los resultados obtenidos y evaluar su relevancia y significado
    \item Presentación de resultados: Redactar y presentar los resultados en forma de informes, artículos científicos o presentaciones
    \item Conclusiones y recomendaciones: Extraer conclusiones basadas en los resultados y ofrecer recomendaciones para futuras investigaciones o aplicaciones prácticas
\end{itemize}

\subsection*{La Hipótesis}
La hipótesis es una premisa que se plantea en forma de pregunta, bajo el cual se quiere comprobar o predecir alguna afirmación o negación; son los supuestos razonados que implican una serie de conceptos, juicios y raciocinios tomados de la realidad estudiada, como una afirmación objetiva sobre una relación entre variables o propiedad de algún fenómeno; que permite ordenar, sistematizar y estructurar el conocimiento.

\subsection*{Observación}
Es notable las normas APA la importancia de la estandarización en la presentación de trabajos académicos. También es crucial resaltar también el valor de la precisión en las citas y referencias para evitar el plagio y mantener la integridad académica. Además, la inclusión de un lenguaje no discriminatorio subraya el compromiso con la diversidad y la equidad.

\subsection*{Comentario}
Es fundamental la metodología de la investigación para garantizar la validez y la confiabilidad de los estudios realizados. La clara formulación del problema, junto con una revisión bastante vasta de la literatura y la adecuada elección de métodos de recolección y análisis de datos, son pasos esenciales que aseguran que los resultados obtenidos sean relevantes y aplicables. La correcta interpretación y presentación de estos resultados contribuyen al avance del conocimiento y a la toma de decisiones informadas en diversas áreas.

\section*{Conclusiones}
\begin{enumerate}
    \item Las Normas APA son fundamentales en el ámbito académico para asegurar la calidad y la integridad de los trabajos escritos. Su correcta aplicación no solo mejora la presentación de los documentos, sino que también protege a los autores y promueve una comunicación clara y efectiva.
    \item La Metodología de la investigación permite describir fenómenos, explicar relaciones, predecir comportamientos y desarrollar teorías basadas en datos y patrones observados. Involucra un proceso sistemático que incluye la formulación del problema, revisión de la literatura, diseño de la investigación, recolección y análisis de datos, interpretación de resultados y presentación de conclusiones.
\end{enumerate}

\section*{Referencias Bibliográficas}
\begin{itemize}
    \item Arteaga, C. y Campos, G. (2004). \textit{Guía para la elaboración de tesis en Trabajo Social}. Serie Metodología y práctica del Trabajo Social. México. UNAM.
    \item Campos, G. (2015). \textit{Tesis}. México. Plaza y Valdés
    \item American Psychological Association. (2020). \textit{Publication manual of the American Psychological Association} (7th ed.). https://doi.org/10.1037/0000165-000
\end{itemize}

\section*{Referencias Electrónicas}
\begin{itemize}
    \item \url{https://www.gob.mx/cms/uploads/attachment/file/133491/\METODOLOGIA_DE_INVESTIGACION.pdf}
    \item 
    \url{https://normas-apa.org/wp-content/uploads/Guia-Normas-APA-7ma-edicion.pdf}
\end{itemize}

\section*{Repositorio Git}
\begin{itemize}
    \item \url{https://github.com/JohnH31/SEMINARIO-DE-TECNOLOG-AS-DE-INFORMACI-N.git}
\end{itemize}

\end{document}
