\documentclass[12pt]{article}
\usepackage{graphicx} % Required for inserting images
\usepackage{amsmath}
\usepackage{url}
\usepackage{amssymb}
\usepackage{fontspec}
\usepackage{cite}
\setmainfont{Times New Roman}
\usepackage[a4paper, margin=2.4cm]{geometry}
\setlength{\parindent}{0pt} % Eliminar sangría
\begin{document}
\begin{center}
\section*{Ciberseguridad, Seguridad Informática y Seguridad de la Información}
HERRERA RODRIGUEZ JONATHAN BENJAMIN
\\SEMINARIO DE TECNOLOGÍAS DE INFORMACIÓN
\\7690-13-1131
\\jbherrerar9@miumg.edu.gt
\end{center}
\section{Resumen}

Son disciplinas interconectadas que buscan proteger la integridad, confidencialidad y disponibilidad de los datos en el entorno digital. La ciberseguridad se centra en la protección contra amenazas externas, como ataques cibernéticos y malware, que pueden comprometer sistemas y redes. La seguridad informática abarca la protección de hardware, software y redes, asegurando que los sistemas informáticos estén resguardados contra el acceso no autorizado y otras amenazas. La seguridad de la información, por su parte, se enfoca en proteger los datos sensibles dentro de una organización, garantizando que solo las personas autorizadas tengan acceso a la información crítica.
Cada una de estas áreas incluye un conjunto de prácticas y herramientas que se implementan para prevenir, detectar y responder a incidentes de seguridad. Por ejemplo, la ciberseguridad incluye la seguridad de red, la seguridad de las aplicaciones y la capacitación del usuario final para evitar errores humanos que puedan comprometer la seguridad. La seguridad informática se preocupa por la integridad, confidencialidad y disponibilidad de la información, aplicando principios como la autenticación y la prevención de pérdida de datos. Finalmente, la seguridad de la información establece protocolos para proteger datos críticos, valiosos y sensibles, adaptándose a las normativas legales y a las necesidades específicas de cada organización.


\section{Palabras Clave}
Ciberataques, Amenazas, Protección de datos, Integridad, Confidencialidad.

\section{Desarrollo del Tema}

\subsection*{Ciberseguridad}
La ciberseguridad es el conjunto de procedimientos y herramientas que se implementan para proteger la información que se genera y procesa a través de computadoras, servidores, dispositivos móviles, redes y sistemas electrónicos.
De acuerdo a los expertos de Information Systems Audit and Control Association (ISACA), la ciberseguridad se define como "una capa de protección para los archivos de información”. También, para referirse a la ciberseguridad, se utiliza el término seguridad informática o seguridad de la información electrónica.

Tal y como indica Kaspersky, la Ciberseguridad es la práctica de defender, con tecnologías o prácticas ofensivas, las computadoras, los servidores, los dispositivos móviles, los sistemas electrónicos, las redes y los datos de ataques maliciosos llevados a cabo por cibercriminales.
Es decir, su objetivo es proteger la información digital en los sistemas interconectados. La Ciberseguridad está comprendida dentro de la Seguridad de la Información. En este enlace podrás conocer las tendencias en Ciberseguridad que marcan (y marcarán) el mundo digital. 
\begin{itemize}
    \item La seguridad de red es la práctica de proteger una red informática de los intrusos, ya sean atacantes dirigidos o malware oportunista.
    \item La seguridad de las aplicaciones se enfoca en mantener el software y los dispositivos libres de amenazas. Una aplicación afectada podría brindar acceso a los datos que está destinada a proteger. La seguridad eficaz comienza en la etapa de diseño, mucho antes de la implementación de un programa o dispositivo.
    \item La seguridad de la información protege la integridad y la privacidad de los datos, tanto en el almacenamiento como en el tránsito.
    \item La seguridad operativa incluye los procesos y decisiones para manejar y proteger los recursos de datos. Los permisos que tienen los usuarios para acceder a una red y los procedimientos que determinan cómo y dónde pueden almacenarse o compartirse los datos se incluyen en esta categoría.
    \item La recuperación ante desastres y la continuidad del negocio definen la forma en que una organización responde a un incidente de ciberseguridad o a cualquier otro evento que cause que se detengan sus operaciones o se pierdan datos. Las políticas de recuperación ante desastres dictan la forma en que la organización restaura sus operaciones e información para volver a la misma capacidad operativa que antes del evento. La continuidad del negocio es el plan al que recurre la organización cuando intenta operar sin determinados recursos.
    \item La capacitación del usuario final aborda el factor de ciberseguridad más impredecible: las personas. Si se incumplen las buenas prácticas de seguridad, cualquier persona puede introducir accidentalmente un virus en un sistema que de otro modo sería seguro. Enseñarles a los usuarios a eliminar los archivos adjuntos de correos electrónicos sospechosos, a no conectar unidades USB no identificadas y otras lecciones importantes es fundamental para la seguridad de cualquier organización.
\end{itemize}
\subsection*{Tipos de ciberamenazas}
Las amenazas a las que se enfrenta la ciberseguridad son tres:
\begin{enumerate}
    \item El delito cibernético incluye agentes individuales o grupos que atacan a los sistemas para obtener beneficios financieros o causar interrupciones.
    \item Los ciberataques a menudo involucran la recopilación de información con fines políticos.
    \item El ciberterrorismo tiene como objetivo debilitar los sistemas electrónicos para causar pánico o temor.
\end{enumerate}

\subsection*{Seguridad Informática}
La seguridad informática o ciberseguridad, es la protección de la información con el objetivo de evitar la manipulación de datos y procesos por personas no autorizadas. Su principal objetivo es que, tanto personas como equipos tecnológicos y datos, estén protegidos contra daños y amenazas hechas por terceros.
La seguridad informática es el conjunto de prácticas, estrategias, métodos, herramientas y procedimientos cuyo objetivo final es garantizar la integridad de los equipos informáticos y de la información que contienen.

\subsection*{¿Cuál es la importancia de la seguridad informática?}
La seguridad informática ha surgido como una necesidad, debido a los intensos cambios en el sector productivo y, a la manera en cómo vive la sociedad mundial gracias a la transformación digital.
Por este motivo, la información se ha convertido en uno de los activos principales de las empresas e individuos y, para mantener sus datos resguardados, deben invertir en este tipo de seguridad.
La seguridad informática se encarga de prevenir y detectar el uso no autorizado de un sistema informático e implica la protección contra intrusos que pretendan utilizar las herramientas y/o datos empresariales maliciosamente o con intención de lucro ilegítimo.

\subsection*{¿Cuáles son los tipos de seguridad informática?}
\begin{enumerate}
    \item Seguridad de hardware: Este tipo de seguridad se relaciona con la protección de dispositivos que se usan para proteger sistemas y redes —apps y programas de amenazas exteriores—, frente a diversos riesgos.

    El método más utilizado es el manejo de sistemas de alimentación ininterrumpida (SAI), servidores proxy, firewall, módulos de seguridad de hardware (HSM) y los data lost prevention (DLP). Esta seguridad también se refiere a la protección de equipos físicos frente a cualquier daño físico.

    \item Seguridad de software: Este tipo de seguridad se emplea para salvaguardar los sistemas frente ataques malintencionados de hackers y otros riesgos relacionados con las vulnerabilidades que pueden presentar los software. A través de estos “defectos” los intrusos pueden entrar en los sistemas, por lo que se requiere de soluciones que aporten, entre otros, modelos de autenticación.
    
    \item Seguridad de red: La seguridad de la red está relacionada con el diseño de actividades para proteger los datos que sean accesibles por medio de la red y que existe la posibilidad de que sean modificados, robados o mal utilizados. Las principales amenazas en esta área son: virus, troyanos, phishing, programas espía, robo de datos y suplantación de identidad.

\end{enumerate}

\subsection*{¿Cuáles son los principios de la seguridad informática?}
Las áreas principales de la información que cubren son 4:
\begin{enumerate}
\item Integridad: Se trata de la autorización de algunos usuarios en particular para el manejo y modificación de datos cuando se considere necesario.

\item Confidencialidad: Únicamente los usuarios autorizados tienen acceso a los distintos tipos de recursos, datos e información, logrando filtrar y robustecer el sistema de seguridad informática.

\item Disponibilidad: Los datos deben estar disponibles para el momento en que sean necesitados. Es la capacidad de permanecer accesible en el sitio, momento y en la forma que los usuarios autorizados lo necesiten.

\item Autenticación: Se basa en la certeza de la información que manejamos.

\end{enumerate}


\subsection*{¿Cómo protegerse de los ciberataques?}
Los computadores y toda la información contenida en ellos puede ser blanco de delincuentes cibernéticos por medio de virus informáticos con los que buscan alterar el funcionamiento del dispositivo y así extraer, dañar o borrar datos.
Hay algunas formas de prevenir estos ciberataques:

\begin{enumerate}
\item No hacer clic en mensajes de redes sociales como Twitter o Facebook que se vean sospechosos o donde haya muchas personas etiquetadas.
\item Cuando no se conozca el remitente de un correo electrónico, es mejor no abrir los archivos adjuntos que esta persona envíe. Primero hay que confirmar la información de la persona que envía, para ahora sí descargar cualquier archivo adjunto.
\item No ingresar a sitios web sospechosos o en los que se ofrecen ganar cosas por hacer clic en ciertas páginas de internet.
\item Evitar descargar música, videos, películas o series gratis de forma ilegal en internet. La gran mayoría de estos sitios que ofrecen descarga de archivos contienen virus que pueden infectar el computador.
\item No hacer clic en anuncios publicitarios sospechosos que se vean falsos en páginas web o que lleguen al correo electrónico.
\item Analizar las memorias USBs, DVDs o CDs antes de ejecutarlos en el computador. Lo importante es mantener actualizados y activos los antivirus en todos los dispositivos que se usen para navegar en internet.
\end{enumerate}

\subsection*{Seguridad de la información}
La seguridad de la información es el conjunto de medidas y técnicas utilizadas para controlar y salvaguardar todos los datos que se manejan dentro de la organización y asegurar que los datos no salgan del sistema que ha establecido la organización. Es una pieza clave para que las empresas puedan llevar a cabo sus operaciones, ya que los datos que maneja son esenciales para la actividad que desarrollan.
De forma mayoritaria, los sistemas de las organizaciones se basan en las nuevas tecnologías, no podemos confundir seguridad de la información y seguridad informática que, si bien están íntimamente relacionadas, no siendo el mismo concepto.
Es importante comprender que cualquier organización, independientemente de su tamaño, cuenta con dato confidenciales, bien de sus clientes, bien de sus trabajadores o bien de ambos, y que por ello tiene que establecer las medidas de seguridad en protección de datos necesarios para garantizar el correcto tratamiento de estos, algo que, con la entrada en vigor primero de LOPD y después de RGPD, no es una opción, sino una obligación.

\subsection*{Objetivos de la seguridad de la información}
Si partimos del hecho de que la seguridad de la información puede cambiar en función de las características de cada organización y del sector al que dedique su actividad económica, podemos hablar de una serie de objetivos comunes que comparten todas las organizaciones del ámbito de la seguridad de la información y la protección de datos.
Encontramos estos objetivos de la seguridad de la información en la norma ISO 27001. La norma establece un modelo para la implementación de sistemas de gestión de seguridad de la información. El principal fin que persigue la norma ISO 27001 es la protección de los activos de información, es decir, equipos, usuarios e información.
Se establece este sistema ISO de seguridad de la información hay que tener en cuenta tres aspectos fundamentales:

\begin{itemize}
    \item Integridad
Los sistemas que gestionan la información tendrán que garantizar la integridad de la misma, es decir, que la información se muestra tal y como fue concebida, sin alteraciones o manipulaciones que no hayan sido autorizadas de forma expresa.
El objetivo principal es garantizar la transmisión de los datos en un entorno seguro, utilizando protocolos seguros y técnicas para evitar posibles riesgos.

\item Confidencialidad
La confidencialidad garantiza que solo las personas o entidades autorizadas tendrán acceso a la información y datos recopilados y que estos no se divulgarán sin el permiso de forma correspondiente. Los sistemas de seguridad de la información tendrán que garantizar que la confidencialidad de la misma no se ve comprometida en ningún momento.

\item Disponibilidad
En este aspecto se garantiza la información que se encuentra disponible en todo momento para todas las personas o entidades autorizadas para su manejo y conocimiento. Para esto deberán existir medidas de soporte y seguridad que se puedan acceder a la información cuando resulte necesario y que evite que se establezcan interrupciones en los servicios.

\end{itemize}

\subsection*{¿Porqué es importante la seguridad de la información en la organización?}
La seguridad de la información se ha convertido en un elemento clave para el funcionamiento de las organizaciones hoy en día, ya que todas ellas manejan datos para poder llevar a cabo su actividad y necesita garantizar su protección e integridad según las leyes vigentes.
Los sistemas de seguridad de la información deben ser capaces de gestionar el riesgo existente y supéralo con el menor impacto para la organización, es decir, tiene que ser capaces de garantizar la resiliencia de la organización y sus sistemas de seguridad con lo que prevenir, evitar y solucionar cualquier riesgo o ataque que se derive del tratamiento de la información y los datos.
Las organizaciones tienen que contar con soluciones tecnológicas adecuadas que no solo aseguren la protección, sino que también permitan conocer en todo momento el estado de dicha protección y que proporcionen las herramientas necesarias para garantizar la continuidad de las organizaciones y su actividad en caso de que sufran un ataque.
3 tipos de información con las que trabajaría cualquier organización
Hay 3 tipos de información con las que trabaja cualquier organización, independientemente de su actividad o sector, y que tiene que ser tenidos en cuenta para realizar a cabo una protección de datos adecuada:


\begin{itemize}
    \item Crítica
La información crítica es la que es indispensable para el correcto funcionamiento de la organización y sus operaciones. La información crítica es la que establece los beneficios de la organización a medio y largo plazo, ya que facilitará las ventas y el servicio al cliente.
Conocer la información y los datos son necesarios para establecer todos los protocolos de seguridad necesarios para su protección.

    \item Valiosa
Es la información que la organización siga adelante. Tiene un alto componente subjetivo y lo que para una organización es información valiosa, para otra puede no serlo, ya que depende de la actividad y el sector.
No toda la información y datos tienen el mismo valor y las empresas deben analizar cuáles son necesarios y cuáles no para el funcionamiento de negocio.

    \item Sensible
La información es sensible en el sentido de que es información privada de los clientes de la organización y, por lo tanto, solo tiene que tener acceso a las mismas personas autorizadas. Los sistemas de seguridad de la información tienen que garantizar la protección de datos de los clientes.

\end{itemize}


\subsection*{Observación}
\begin{enumerate}
    \item Aunque se mencionan varias amenazas y prácticas de protección, podría ser útil incluir ejemplos específicos de incidentes de ciberseguridad conocidos para ilustrar la importancia de las medidas de seguridad, ya que los ejemplos claros podrían ser también utilizados por malvados se encuentra pocos ejemplos de como ejercer en esta área.
    \item Nos enfocamos a gran medida en amenazas externas, pero podría ser útil abordar también los riesgos internos, como el mal manejo de la información por parte de empleados, que es una causa común de brechas de seguridad.
    \item En un mundo cada vez más digitalizado, la ciberseguridad se ha convertido en una prioridad crucial para proteger tanto a individuos como a organizaciones de una variedad de amenazas cibernéticas. La implementación de prácticas robustas de ciberseguridad es esencial para salvaguardar la integridad de la información y mantener la confianza en los sistemas digitales.
\end{enumerate}

\subsection*{Comentario}
\begin{enumerate}
    \item La organización de lo investigado y lo claro facilita la comprensión de cada tema. Cada sección trata aspectos específicos, como tipos de amenazas, áreas de seguridad, y principios clave, lo cual enriquece el análisis.
    \item Se incluyen ejemplos concretos y consejos prácticos sobre cómo protegerse de ciberataques, lo que agrega valor a la investigación al ofrecer no solo teoría, sino también aplicaciones prácticas.
\end{enumerate}

\section*{Conclusiones}
\begin{enumerate}
    \item En el tema de ciberseguridad es esencial para proteger la infraestructura digital contra ataques maliciosos. Su enfoque debe ser integral, cubriendo desde la protección de redes y aplicaciones hasta la educación de los usuarios finales, ya que cada capa de defensa contribuye a mitigar los riesgos en un entorno digital cada vez más amenazado.
    \item Metiéndonos un poco a lo que es la seguridad informática abarca tanto la protección física del hardware como la seguridad lógica del software. Es fundamental implementar controles técnicos, como firewalls y sistemas de detección de intrusos, para garantizar que los sistemas y los datos permanezcan seguros frente a accesos no autorizados y vulnerabilidades internas y externas.
    \item Hablando sobre seguridad de la información vemos que va más allá de la tecnología, abarcando la gestión de datos sensibles y la implementación de políticas que aseguren la confidencialidad, integridad y disponibilidad de la información. Las organizaciones deben adoptar marcos normativos, Se ve mucho el tema de la calidad en cuanto a este tema de seguridad de la información vemos la ISO 27001, para estructurar sus sistemas de gestión de seguridad de la información de manera efectiva.
    \item La seguridad informática busca la preservación de la confidencialidad, integridad y disponibilidad de la información. Debido a que la información corporativa es uno de los activos más importantes que maneja toda empresa, es importante invertir en un sistema de gestión que busque garantizar su protección.
\end{enumerate}

\section*{Referencias Bibliográficas}
\begin{itemize}
    \item ISO/IEC 27001:2013. \textit{Information technology — Security techniques — Information security management systems — Requirements}. International Organization for Standardization.
    \item Stallings, W. (2017). \textit{Cryptography and Network Security: Principles and Practice}. Pearson.
\end{itemize}

\section*{Referencias Electrónicas}
\begin{itemize}
    \item \url{https://www.pmg-ssi.com/2021/03/que-es-la-seguridad-de-la-informacion-\ y-cuantos-tipos-hay/}    
    \item \url{https://latam.kaspersky.com/resource-center/definitions/what-is-cyber-security}
    \item \url{https://www.infosecuritymexico.com/es/ciberseguridad.html#:~:text=La%20 \ ciberseguridad%20es%20el%20conjunto,m%C3%B3viles%2C%20redes%20y%20sistemas%20electr%C3%B3nicos.}
    \item \url{https://www.lisainstitute.com/blogs/blog/diferencia-ciberseguridad-seguridad-informatica-seguridad-informacion}

        \item \url{https://www.ucatalunya.edu.co/blog/seguridad-informatica-la-importancia-y-\lo-que-debe-saber#:~:text=La%20seguridad%20inform%C3%A1tica%20o%20ciberseguridad,\procesos%20por%20personas%20no%20autorizadas.}
\end{itemize}

\section*{Repositorio Git}
\begin{itemize}
    \item \url{https://github.com/JohnH31/SEMINARIO-DE-TECNOLOG-AS-DE-INFORMACI-N.git}
\end{itemize}

\end{document}
