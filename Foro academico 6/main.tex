\documentclass[12pt]{article}
\usepackage{graphicx} % Required for inserting images
\usepackage{amsmath}
\usepackage{url}
\usepackage{amssymb}
\usepackage{fontspec}
\usepackage{cite}
\setmainfont{Times New Roman}
\usepackage[a4paper, margin=2.4cm]{geometry}
\setlength{\parindent}{0pt} % Eliminar sangría
\begin{document}
\begin{center}
\section*{Salud digital y Ergonomía digital}
HERRERA RODRIGUEZ JONATHAN BENJAMIN
\\SEMINARIO DE TECNOLOGÍAS DE INFORMACIÓN
\\7690-13-1131
\\jbherrerar9@miumg.edu.gt
\end{center}
\section{Resumen}

Hablando de la Salud digital y la ergonomía digital están transformando el sector sanitario mediante el uso de tecnologías innovadoras. La eSalud como se hace llamar tembien salud digital, aplica las TIC para mejorar la atención médica, permitiendo una gestión más eficiente y un acceso más equitativo. Servicios como la telemedicina, las apps de salud, los wearables y la realidad aumentada facilitan la monitorización de pacientes, el acceso a información médica y la toma de decisiones de los profesionales. Además, tecnologías como el Internet de las Cosas (IoT), Big Data e Inteligencia Artificial personalizan tratamientos y mejoran la eficiencia hospitalaria.
Extendiendonos a lo que es la ergonomía digital se enfoca en cómo las personas interactúan con la tecnología, optimizando la experiencia de usuario (UX) en interfaces digitales. Esto incluye garantizar la accesibilidad para todos los usuarios, independientemente de sus capacidades, y facilitar el acceso a la información mediante interfaces intuitivas. Factores como la compatibilidad entre dispositivos, el diseño visual y la reputación online son fundamentales para crear entornos digitales eficientes y seguros.
estas áreas fomentan una atención médica más accesible y personalizada, mientras mejoran la calidad de la experiencia digital para pacientes y profesionales. La salud digital y la ergonomía, combinadas, apuntan a un futuro donde la tecnología y el diseño interactivo proporcionen una sanidad más inclusiva, eficiente y adaptada a las necesidades de cada usuario.

\section{Palabras Clave}
Telemedicina, Inteligencia Artificial, Wearables, UX y TIC.

\section{Desarrollo del Tema}

\subsection*{Salud Digital}
El sector de la sanidad está viviendo una auténtica revolución digital con la inclusión de la tecnología en sus procesos. La salud digital o eSalud —traducción del inglés eHealth— se refiere al uso de las TIC en el sector sanitario para dotarlo de recursos innovadores que permitan una gestión más eficiente y un diagnóstico más óptimo, en definitiva, una mejor atención a los pacientes. Esto incluye innovaciones tanto en la comunicación médico-paciente como en la investigación o la gestión hospitalaria, entre otros.
La eSalud es una industria en expansión que en 2018 invirtió, a nivel global, 14.600 millones de dólares, según el portal de datos Statista, lo que supuso un 1.200 \% más que en 2010. El interés por la salud digital alcanza al 58 \% de los países miembro de la Organización Mundial de la Salud (OMS), dado que cuentan con estrategias específicas para la digitalización de la salud.

\subsection*{Servicios de la salud digital}
La versatilidad que ofrecen las nuevas tecnologías es también una de las mejores cualidades en su aplicación para el ámbito sanitario. Detallamos los servicios más conocidos:

\begin{itemize}
    \item Telemedicina. Este sistema facilita las consultas a distancia, es decir, permite la atención sanitaria a personas situadas en lugares remotos o con un limitado acceso a la sanidad. Además, supone un ahorro de tiempo, costes y desplazamientos tanto para médicos como para pacientes.
    \item Apps. Gracias a las aplicaciones móviles dedicadas a la salud podemos convertir nuestro smartphone en un entrenador personal, en un monitor de sueño, en un diagnosticador, etc. Las hay tanto para profesionales sanitarios como para pacientes.
    \item Serious Games. Este tipo de videojuegos se utilizan como recursos educativos para sanitarios y estudiantes con el objetivo de facilitar su formación. También los hay para aquellas personas que deseen ampliar conocimientos sobre determinadas patologías.
    \item Tecnología vestible. Más conocidos como wearables, incluyen ropa y complementos inteligentes, como pulseras, gafas y relojes, que monitorizan y recaban datos sobre nuestra salud y condición física.
    \item Realidad aumentada. Sirve al personal sanitario para, por ejemplo, visualizar órganos en 3D o consultar el historial del paciente en tiempo real. Incluso durante una intervención quirúrgica a través de unas gafas especialmente ideadas para la realidad aumentada.
    \item Historia clínica electrónica. La digitalización de nuestro historial médico permite centralizar la información, de forma que el paciente pueda compartirlo de forma segura y el médico acceder a él en cualquier momento.
\end{itemize}

\subsection*{Ventajas y beneficios de la eSalud}
La salud digital, gracias a servicios como los mencionados anteriormente, permite aplicar nuevos métodos, medios, herramientas y canales que repercuten en una serie de beneficios:

Mejora la monitorización de los pacientes

El contacto es más directo al abrirse una vía de comunicación digital que reduce la distancia médico-paciente. La tecnología también permite monitorizar el estado de salud de los pacientes para registrar su evolución en tiempo real.

Los pacientes cuentan con más información

Al tener un mayor conocimiento y gestión sobre su propia salud, los pacientes pueden tomar mejores decisiones al respecto. También permite el acceso a manuales y buenas prácticas a través de las TIC, algo muy útil, por ejemplo, durante una pandemia si las fuentes son de confianza.
   
   Ayuda a adquirir hábitos más saludables

Las nuevas tecnologías están modificando la manera en que nos cuidamos, ya sea registrando lo que comemos, el ejercicio físico que hacemos o monitorizando el sueño o la frecuencia cardíaca a través de aplicaciones y otros dispositivos tecnológicos.

   Facilita la toma de decisiones del personal sanitario

El uso de la salud digital también está transformando el manejo de las enfermedades por parte de los profesionales. La tecnología permite, por ejemplo, identificar más fácilmente las intervenciones óptimas o la detección precoz de enfermedades.

   Fomento de una sanidad más accesible y equitativa

El acceso a los servicios sanitarios se vuelve independiente del tiempo y el espacio, y evita desplazamientos innecesarios. Además, acerca la salud a más gente, sobre todo a enfermos en riesgo de exclusión, generando mayor igualdad de oportunidades.
  
   Mejora la eficiencia de los hospitales y los centros de salud

La conectividad de estas instalaciones agiliza el funcionamiento del sistema sanitario, minimiza el margen de error humano y disminuye los costes. Además, técnicas como el big data permiten automatizar procesos.

\subsection*{Nuevas tecnologías en la salud}
La digitalización de la salud contempla el uso de innovadoras tecnologías. Repasamos algunas de ellas y sus aplicaciones concretas:
\begin{itemize}
    \item Internet of Things (IoT)
El Internet de las Cosas permite personalizar la atención sanitaria, ahorrar costes y reducir los errores de diagnóstico y los tiempos de espera. La conexión entre el mundo físico y el digital en instrumentos como inhaladores o audímetros, entre otros, será fundamental.

    \item Big data
El análisis de macrodatos a través del big data permite personalizar tratamientos y detectar factores de riesgo y posibles efectos secundarios de los fármacos. Esta tecnología fue decisiva para conocer y controlar mejor la expansión de la COVID-19.

    \item Inteligencia Artificial
La Inteligencia Artificial facilita la toma de decisiones exhaustivas a los sanitarios para así ofrecer los mejores tratamientos. Durante la crisis del coronavirus sirvió para identificar la secuencia de anticuerpos compatibles con futuras terapias.

    \item Blockchain
El blockchain permite un acceso seguro a los historiales de los pacientes, lo que repercute en una mayor eficiencia administrativa. En ese sentido, los laboratorios farmacéuticos también pueden llevar un registro más preciso en la fabricación de medicamentos.

    \item Impresión 3D y 4D
El uso de la impresión 4D en las ecografías permite, por ejemplo, conocer con mayor precisión el desarrollo estructural y funcional del sistema nervioso del feto. Por otro lado, la escasez de material sanitario durante la crisis del coronavirus inspiró la fabricación de piezas sanitarias con impresoras 3D.

    \item Chatbots
Los chatbots son una herramienta que facilita la comunicación médico-paciente al hacerla más rápida y directa. La OMS habilitó uno de estos canales durante la pandemia de COVID-19.

    \item Realidad virtual
Los principales usos que permite una tecnología como la realidad virtual incluyen la formación práctica de profesionales sanitarios, la rehabilitación de pacientes y el tratamiento de trastornos psicológicos.

\end{itemize}


\cite{iberdrolaEsalud}, \cite{sanofiSaludDigital}

\subsection*{Qué es la Ergonomía Digital}
La ergonomía digital se refiere a la forma en que los usuarios interactúan con las tecnologías, como computadoras y dispositivos móviles. Estudia cómo el diseño de interfaces gráficas (GUI) puede mejorar la eficiencia y satisfacción del usuario al trabajar con un sistema informático.

Así, la Ergonomía Digital se compone de un conjunto de técnicas y conocimientos que facilitan la adaptación de Internet a la capacidad y necesidades de las personas, de manera que mejore la eficacia, la eficiencia, el acceso, la confianza y la seguridad garantizando así una óptima experiencia a los usuarios en el entorno virtual.

El objetivo principal de la Ergonomía Digital es lograr una experiencia intuitiva para los usuarios sin sacrificar ninguna función o calidad técnica. La ergonomía digital toma en cuenta muchos factores diferentes para crear entornos confortables e interactivos que permiten a los usuarios realizar sus actividades de manera fácil y rápida.
Algunas áreas importantes que influyen en la ergonomía digital incluyen:

\begin{itemize}
    \item Personas
        \begin{itemize}
            \item Aptitudes
            \item Contexto
            \item Objetivos
        \end{itemize}
    \item Tecnología
        \begin{itemize}
            \item Estándares
            \item Compatiblidad
            \item Usabilidad
            \item Visibilidad
            \item Seguridad
            \item Accesibilidad visual, auditiva u otros requisitos especiales
        \end{itemize}
    \item Experiencia de Uso:
        \begin{itemize}
            \item Facilidad de aprendizaje
            \item Diseño centrado en el usuario
            \item Estructura general del sitio web o software programado para cumplir con las necesidades particulares del usuario final.
        \end{itemize}
    
\end{itemize}

\subsection*{Accesibilidad}
La accesibilidad consiste en garantizar el acceso a la información de una web y la correcta interacción con la misma, a todas las personas y bajo cualquier circunstancia.
La Accesibilidad es una parte importante de la Ergonomía Digital, ya que garantiza el acceso y uso eficiente para todos los usuarios. Esto significa diseñar interfaces digitales con estándares de alta calidad en cuanto a su facilidad de navegación, comprensión y operatividad. Al hacerlo se logra un entorno donde todos los usuarios puedan tener igualdad al interactuar con dichas plataformas sin importar sus limitaciones físicas o cognitivas. Por lo tanto, no solo mejora la experiencia del usuario sino que además contribuye a promover inclusión social.

\subsection*{Experiencia de usuario}
La Experiencia de Usuario es la sensación, sentimiento, respuesta emocional, valoración y satisfacción del usuario respecto a un
website como resultado del fenómeno de la interacción con él.
La Experiencia de Usuario (también llamada UX) es una parte clave de la Ergonomía Digital. Esto significa que el diseño y desarrollo web deben ser realizados con el objetivo final del usuario en mente, para garantizar una experiencia óptima para los visitantes a su sitio web. El UX incluye todos los aspectos relacionados con la interacción entre el usuario y su dispositivo digital, así como también las estrategias necesarias para mejorar la eficacia y satisfacción del usuario durante sus sesiones online.
Los elementos fundamentales que se consideran son:

\begin{itemize}
    \item contenido accesible
    \item facilidad de navegación intuitiva
    \item interfaz amigable al usuario
    \item integración correcta de tecnologías avanzadas como IA o Machine Learning.
\end{itemize}
Al optimizarse estos factores dentro del proyecto web podemos lograr crear productos exitoso dirigidos hacía nuestros clientes finales sin sacrificar calidad ni funcionalidad.

\subsection*{Visibilidad}
Se refiere al buen posicionamiento de un sitio web o, lo que es lo mismo, la tarea de ajustar la información de las páginas que se pretenden hacer aparecer en primeras posiciones de los resultados de los buscadores más conocidos, así como la encontrabilidad (findability) de la información que contiene.

\subsection*{Compatibilidad}
Una web móvil es aquella en la que el usuario puede acceder a la información desde cualquier lugar, independientemente del tipo de dispositivo que utilice para ello.
Para que un sitio web sea compatible con la mayoría de sistemas y navegadores, hay varios aspectos a tener en cuenta a la hora de desarrollarlo.

\begin{itemize}
    \item Respetar los estándares web actuales para garantizar su funcionamiento correcto en diferentes plataformas.
    \item Hacer pruebas exhaustivas usando diferentes versiones de navegadores y dispositivos móviles para comprobar el rendimiento general del sitio.
    \item Probar la compatibilidad retroactiva con versiones anteriores del navegador para ofrecer a los usuarios una experiencia óptima independientemente de qué versión utilicen
    \item Optimizar el contenido multimedia (por ejemplo imágenes) al diseñar páginas webs modernas que permitan mejor adaptabilidad entre dispositivos y pantallas diversificadas
\end{itemize}

\subsection*{Diseño visual}
El diseño visual es una disciplina profesional que estudia los sistemas de información con el objeto de convertir los datos en formas visuales, teniendo en cuenta los procesos perceptivos. Consiste en la creación de imágenes funcionales con fines
netamente comunicacionales.

\subsection*{Reputación y Confianza Online}
Consiste en el grado de veracidad y credibilidad que refleja el contenido y la información que proporcionan una web. Esta confianza y reputación ha de garantizar la no adulteración de la información en cuanto a su interpretación por parte de los usuarios.

\subsection*{Internacionalización}
La internacionalización (I18n) es el proceso de diseño y desarrollo de un website que permite una fácil localización con destino a audiencias de diferentes culturas, regiones o idiomas. Se trata de técnicas y pautas de diseño y
desarrollo que facilitan esa migración en el futuro, pero que también pueden tener una utilidad considerable aunque la localización jamás se produzca.


\subsection*{Otros aspectos y características de la Ergonomía Digital a tener en cuenta}
\begin{itemize}
    \item Es un concepto innovador: es la primera vez que se agrupa este conjunto de conocimientos.
    \item Genera un nueva ciencia multidisciplinar.
    \item Crea un nuevo perfil profesional: Ergónomo Digital.
    \item Requiere formación continua.
    \item Proporciona un servicio de impacto adaptable ya que permite ofrecer servicios independientes.
    \item Propicia un nuevo marco de trabajo en I+D+i.
\end{itemize}

\cite{spriErgonomiaDigital}


\subsection*{Observación}
\begin{enumerate}
    \item La salud digital y la ergonomía digital están transformando la manera en que interactuamos con la tecnología en el ámbito sanitario, mejorando tanto la accesibilidad como la eficiencia de los servicios médicos. 
    \item  Uno de los temas a dar enfoque son las nuevas tecnologías como el Internet de las Cosas y la Inteligencia Artificial están permitiendo un diagnóstico más preciso y personalizado, revolucionando el seguimiento y tratamiento de los pacientes.
    \item Dando paso a la accesibilidad digital es fundamental en la ergonomía para garantizar que todos los usuarios, independientemente de sus limitaciones físicas o cognitivas, puedan interactuar con las plataformas de manera eficaz y segura.
\end{enumerate}

\subsection*{Comentario}
\begin{enumerate}
    \item El estar metiendo las tecnologías como la realidad aumentada y los wearables en el campo de la salud está proporcionando nuevas formas de seguimiento y diagnóstico. Estas innovaciones están facilitando el acceso remoto y la atención personalizada, algo que en muchos casos reduce las barreras geográficas y temporales entre médicos y pacientes, beneficiando especialmente a quienes viven en áreas rurales o aisladas.
    \item Escuchando a la ergonomía digital se convierte en un componente esencial en el desarrollo de tecnologías médicas, ya que el diseño intuitivo y accesible puede marcar la diferencia en la aceptación y uso de estas herramientas. En un contexto donde las tecnologías avanzadas están cada vez más presentes, la facilidad de uso y la experiencia del usuario son factores clave para asegurar el éxito y la adopción masiva de soluciones tecnológicas.
\end{enumerate}

\section*{Conclusiones}
\begin{enumerate}
    \item Vemos que a medida que la tecnología sigue avanzando, es crucial que tanto los desarrolladores como los profesionales de la salud se centren en el diseño centrado en el usuario, asegurando que las soluciones digitales sean accesibles para todos. Esta sinergia entre tecnología y ergonomía será vital para el futuro del cuidado de la salud.
    \item La escabilidad en que las tecnologías emergentes van evolucionando, como el Internet de las Cosas y la Inteligencia Artificial, está redefiniendo la relación entre los pacientes y el sistema de salud, promoviendo un enfoque más preventivo y personalizado. Esto requiere una integración fluida entre los sistemas médicos y los dispositivos tecnológicos para maximizar su impacto.


\end{enumerate}

\bibliographystyle{plain}
\bibliography{referencias}

\section*{Repositorio Git}
\begin{itemize}
    \item \url{https://github.com/JohnH31/SEMINARIO-DE-TECNOLOG-AS-DE-INFORMACI-N/tree/main/Foro%20academico%205}
\end{itemize}

\end{document}
